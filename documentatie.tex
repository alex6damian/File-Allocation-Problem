\documentclass[12pt, a4paper]{article}

% --- PACHETE NECESARE ---
\usepackage[romanian]{babel}  % Setări pentru limba română
\usepackage[utf8]{inputenc}   % Encoding
\usepackage{geometry}         % Margini
\usepackage{amsmath}          % Formule matematice
\usepackage{graphicx}         % Imagini și grafice
\usepackage{hyperref}         % Link-uri și cuprins interactiv
\usepackage{listings}         % Pentru cod sursă
\usepackage{xcolor}           % Culori pentru cod
\usepackage{float}            % Poziționare figuri

% --- SETĂRI PAGINĂ ---
\geometry{left=2.5cm, right=2.5cm, top=2.5cm, bottom=2.5cm}

% --- CONFIGURARE COD SURSĂ (Go) ---
\definecolor{codegreen}{rgb}{0,0.6,0}
\definecolor{codegray}{rgb}{0.5,0.5,0.5}
\definecolor{backcolour}{rgb}{0.95,0.95,0.92}

\lstdefinelanguage{Go}{
  keywords={break, default, func, interface, select, case, defer, go, map, struct, chan, else, goto, package, switch, const, fallthrough, if, range, type, continue, for, import, return, var},
  keywordstyle=\color{blue}\bfseries,
  comment=[l]{//},
  commentstyle=\color{codegreen}\ttfamily,
  stringstyle=\color{red}\ttfamily,
  morestring=[b]',
  morestring=[b]"
}

\lstset{
    backgroundcolor=\color{backcolour},   
    basicstyle=\ttfamily\footnotesize,
    keywordstyle=\color{blue},
    commentstyle=\color{codegreen},
    breaklines=true,                 
    captionpos=b,                    
    keepspaces=true,                 
    numbers=left,                    
    numbersep=5pt,                  
    tabsize=4,
    frame=single
}

% --- DATELE PROIECTULUI ---
\title{\textbf{Abordare Microeconomică pentru Alocarea Optimă a Resurselor în Sisteme Distribuite}}
\author{
    \textbf{Studenți:} \\
    Damian Alexandru (342) \\
    Hornet Alex (342) \\
    Opran Andrei (342) \\
    \\
    \textbf{Materie:} Sisteme Distribuite \\
    \textbf{Profesor:} [Conf. Dr. Andrei Patrascu]
}
\date{}

\begin{document}

\maketitle
\tableofcontents
\newpage

% =========================================================================
% 1. INTRODUCERE (Cerință: context, descriere problemă) [cite: 653]
% =========================================================================
\section{Introducere}

% TODO: Scrie aici despre Problema Alocării Fișierelor (FAP).
% Idei de atins:
% - Context: Sistemele distribuite moderne trebuie să împartă resurse (fișiere, baze de date).
% - Problema: Cum împărțim un fișier pe N noduri astfel încât să fie accesat rapid?
% - Trade-off-ul: Dacă punem totul pe un nod, crește timpul de așteptare (coada). Dacă fragmentăm prea mult, crește costul de comunicare.
% - Soluția propusă: O abordare microeconomică, unde nodurile "negociază" resursele pentru a minimiza costul global.

Arhitecturile  distribuite reprezinta solutia optima pentru gestionarea volumelor mari de date si a sarcinilor computationale complexe.
Un sistem distribuit poate fi vazut ca un set de agenti de calcul interconectati care partajeaza resurse pentru a indeplini sarcini comune.
Partajarea resurselor aduce beneficii din punct de vedere al performantei, dar necesita un sistem eficient pentru alocarea si accesul la resurse.

Proiectul nostru abordeaza \textbf{Problema Alocării Fișierelor (File Allocation Problem - FAP)}, ce consta in deteminarea unui mod optim de a imparti
o resursa divizibila (de exemplu un fisier cu dimensiune mare, sau o baza de date) pe un set de noduri. Obiectivul final este de a maximiza performanta 
sistemului distribuit, performanta fiind masurata prin minimizarea unui cost total.

In principal, procesul de alocare a resurselor trebuie sa aiba in vedere doi factori, care daca ar fi luati individual, ar rezulta in strategii opuse:
\begin{itemize}
    \item \textbf{Costul de comunicare:} Minimizarea traficului se face optim prin plasarea resursei nedivizate pe nodul care o solicita cel mai frecvent, 
    sau pe un nod central cu costul de comunicare minimal.
    \item \textbf{Timpul de procesare:} Plasarea a mai multor resurse pe un singur nod rezulta in cozi de asteptare si aglomerarea procesorului, crescand timpul
    de raspuns. Astfel, ajungem la concluzia ca trebuie sa distribuim uniform resursa pe toate nodurile.
\end{itemize}

In acest proiect se propune si se analizeaza o abordare \textbf{microeconomica} pentru rezolvarea problemei. Astfel, fiecare nod este considerat un agent
economic dintr-o piata, care ia decizii bazate pe impactul (costul sau beneficiul) preluarii unei cantitati de resursa. Sistemele centralizate se scaleaza
greu, si pot deveni puncte de esec. Algoritmii descentralizati din aceasta lucrare, beneficiind de resursele distribuite ale sistemului, permit nodurilor
sa negocieze si sa migreze fractiuni din resursa pentru a atinge un echilibru economic in tot setul de noduri, minimizand costul de comunicare si
timpul de procesare.

% =========================================================================
% 2. DESCRIEREA ALGORITMULUI SERIAL (Cerință: algoritm serial) [cite: 654]
% =========================================================================
\section{Descrierea Algoritmului Serial}

% TODO: Descrie modelul matematic centralizat (Cum ar rezolva un singur calculator problema).
% 1. Definește notațiile: lambda (rata sosire), mu (rata servire), x_i (fracțiunea alocată).
% 2. Scrie formula de Cost Global (Eq 1 din PDF-ul teoriei):
%    C = Suma( (Ci + K / (mu - lambda*xi)) * xi )
% 3. Explică faptul că un algoritm serial ar calcula derivatele parțiale pentru toate nodurile și ar rezolva sistemul de ecuații centralizat.



Problema optimizării poate fi formulată matematic astfel:
\begin{equation}
    C = \sum_{i=1}^{N} \left( C_i + \frac{K}{\mu - \lambda_i x_i} \right) x_i
\end{equation}
unde obiectivul este minimizarea $C$ sub constrângerea $\sum x_i = 1$.


% =========================================================================
% 3. DESCRIEREA VARIANTEI DISTRIBUITE (SPMD) [cite: 655]
% =========================================================================
\section{Descrierea Variantei Distribuite (SPMD)}

% TODO: Explică transformarea în algoritm distribuit.
% - Modelul SPMD (Single Program, Multiple Data): Fiecare nod rulează același cod (structura Node din Go).
% - Descrie pașii unei iterații:
%   1. Calcul Local: Fiecare nod își calculează Utilitatea Marginală (Derivata I) și, opțional, factorul de scalare (Derivata II).
%   2. Comunicare: Schimb de mesaje (Broadcast sau Pairwise) pentru a afla starea celorlalți.
%   3. Actualizare: Ajustarea variabilei locale x_i.

Implementarea utilizează modelul SPMD, unde fiecare nod este o instanță independentă care execută următorul ciclu:

\begin{lstlisting}[language=Go, caption={Structura nodului în implementare}]
type Node struct {
    ID         int
    Lambda     float64 // Trafic
    Allocation float64 // Resursa x_i
}
// Fiecare nod ruleaza intr-un Goroutine separat
\end{lstlisting}

% Aici poți detalia cei 3 algoritmi implementați (First Derivative, Second Derivative, Pairwise).


% =========================================================================
% 4. ANALIZĂ TEORETICĂ ȘI DISCUȚII [cite: 656]
% =========================================================================
\section{Analiză Teoretică și Discuții}

% 4.1 Corectitudine [cite: 657]
\subsection{Analiza Corectitudinii}
% TODO: Discută premisele (ex: funcția de cost convexă).
% Menționează proprietățile de Fezabilitate (suma x_i rămâne 1) și Monotonie (costul scade mereu).

% 4.2 Complexitate [cite: 658]
\subsection{Analiza Complexității}
% TODO:
% - Algoritm 1 & 2: Necesită coordonare globală (O(N) mesaje per iterație sau broadcast).
% - Algoritm Pairwise: Comunicare locală (O(1) per pereche), dar convergență posibil mai lentă în număr de pași.

% 4.3 Scalabilitate (Avantaje/Dezavantaje) [cite: 659]
\subsection{Variația Numărului de Noduri și Scalabilitate}
% TODO: Discută ce se întâmplă când N crește.
% Pairwise este mai scalabil pentru N mare. Newton (Alg 2) e rapid pentru N mic/mediu.

% 4.4 Exemple Topologii [cite: 659]
\subsection{Exemplificare pe Topologii}
% TODO: Adaugă graficele generate sau screenshot-uri din consolă.
% Compară rezultatele pentru topologia Fully Connected vs. Ring (sau ce ai testat în Pairwise).

\begin{figure}[H]
    \centering
    % \includegraphics[width=0.8\textwidth]{grafic_convergenta.png}
    \caption{Evoluția costului global în funcție de iterații pentru cei 3 algoritmi.}
\end{figure}


% =========================================================================
% 5. CONCLUZII
% =========================================================================
\section{Concluzii}
% TODO: Sumarizează ce ai demonstrat (Algoritmul 2 e cel mai rapid, Pairwise e cel mai robust).


% =========================================================================
% BIBLIOGRAFIE
% =========================================================================
\begin{thebibliography}{9}
\bibitem{kurose}
  J. F. Kurose și R. Simha,
  \textit{A Microeconomic Approach to Optimal Resource Allocation in Distributed Computer Systems},
  IEEE Transactions on Computers, Vol. 38, No. 5, May 1989.

\bibitem{implementare}
  Sursă cod proiect: \texttt{resource\_allocation.go}.
\end{thebibliography}

\end{document}